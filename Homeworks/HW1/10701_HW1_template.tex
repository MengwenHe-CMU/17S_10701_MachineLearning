\documentclass{article}

\usepackage[usenames,dvipsnames]{color}
\usepackage{bm}
\usepackage{amsmath}
\usepackage{amssymb}
\usepackage{graphicx}
\usepackage[colorlinks=true,urlcolor=blue]{hyperref}
\usepackage{geometry}
\geometry{margin=1in}
\usepackage{float}
\setlength{\marginparwidth}{2.15cm}
\usepackage{booktabs}
\usepackage{epsfig}
\usepackage{setspace}
\usepackage{parskip}
\usepackage[]{algorithm2e}
\usepackage{comment}
\usepackage{pdfpages}
%\usepackage{physics}
\usepackage{enumerate}

%\newcommand{\comment}[1]{\textcolor{blue}{\textsc{\textbf{[#1]}}}}

 \newenvironment{soln}{
     \leavevmode\color{blue}\ignorespaces
 }{}

\makeatletter
\newcommand{\removelatexerror}{\let\@latex@error\@gobble}
\makeatother

\begin{document}

\section*{}
\begin{center}
  \centerline{\textsc{\LARGE Homework 1}}
  \vspace{0.5em}
  \centerline{\textsc{\Large MLE, MAP estimates; Linear and Logistic Regression}}
  \vspace{1em}
  \textsc{\large CMU 10-701: Machine Learning (Spring 2017)} \\
  \vspace{1em}
  \centerline{OUT: Jan 31}
  \centerline{DUE: Feb 10, 11:59 PM}
  \centerline{NAME: Mengwen He}
  \centerline{ADREW ID: mengwenh}
	
\end{center}

\section*{Part A: Multiple Choice Questions}

\begin{enumerate}
	\item For each case listed below, what type of machine learning problem does it belong to?
	\begin{enumerate}
		\item Advertisement selection system, which can predict the probability whether a customer will click on an ad or not based on the search history. \\
		\textbf{Answer:}
		\underline{B. Supervised learning: Regression}\\
		A task, with ads click statistics and search history as input data, outputs the prediction of continuous probability of clicking an ad.
		\item U.S post offices use a system to automatically recognize handwriting on the envelope. \\
		\textbf{Answer:}
		\underline{A. Supervised learning: Classification}\\
		A task, with handwriting samples and their labels as input data, outputs the prediction of discrete numbers/letters of a handwriting on the envelope.
		\item Reduce dimensionality using principal components analysis (PCA). \\
		\textbf{Answer:}
		\underline{C. Unsupervised learning}\\
		A task, without training data as input, outputs a description of reduced dimensionality. 
		\item Trading companies try to predict future stock market based on current market conditions. \\
		\textbf{Answer:}
		\begin{itemize}
			\item \underline{A. Supervised learning: Classification}\\
			A task, with current market conditions as input, outputs the prediction of discrete stock market conditions, say bull or bear market.
			\item \underline{B. Supervised learning: Regression}\\
			A task, with current market conditions as input, outputs the prediction of continuous stock market conditions, say stock price.
		\end{itemize}
		
		\item Repair a digital image that has been partially damaged. \\
		\textbf{Answer:}
		\begin{itemize}
			\item \underline{A. Supervised learning: Classification}\\
			A task, with digital images database as input, outputs the prediction of discrete pixel value in the damaged zone.
			\item \underline{B. Supervised learning: Regression}\\
			A task, with digital images database as input, outputs the prediction of continuous parameters of a color distribution model to form a discrete patch to cover the damaged zone.
			\item \underline{C. Unsupervised learning}\\
			A task, without training data as input, outputs a description of a damaged pixel according to its surrounding pixel values, e.g. interpolation or extrapolation.
		\end{itemize}
	\end{enumerate}
	Type of machine learning problem:
	\begin{itemize}
		\item [A.] Supervised learning: Classification
		\item [B.] Supervised learning: Regression
		\item [C.] Unsupervised learning
	\end{itemize}
	
	\item For four statements below, which one is wrong?
	\begin{itemize}
		\item [A.] In maximum a posterior (MAP) estimate, data overwhelms the prior if we have enough data.
		\item [B.] There are no parameters in non-parametirc models.
		\item [C.] $P(X \cap Y \cap Z)=P(Z|X \cap Y) P(Y|X) P(X)$.
		\item [D.] Compared with parametric models, non-parameter models are flexible, since they don't make strong assumptions.
	\end{itemize}
	\textbf{Answer:}
	\underline{B. There are no parameters in non-parametirc models.} is wrong.\\
	Non-parametric model still needs parameters to describe the model, but the number of model's parameters is not fixed and will grow with the data size. The non-parametric only means that there is weak assumption on the model's type defined by a fixed number of parameters.
	
	\item There are about $12\%$ people in U.S. having breast cancer during their lifetime. One patient has a positive result for the medical test. Suppose the sensitivity of this test is $90\%$, meaning the test will be positive with probability $0.9$ if one really has cancer. The false positive is likely to be $2\%$. Then what is the probability this patient actually having cancer based on Bayes Theorem?\\
	A. $90\%$ \hspace{0.1\textwidth} B. $86\%$ \hspace{0.1\textwidth} C. $12\%$ \hspace{0.1\textwidth} D. $43\%$\\
	\textbf{Answer:}
	\underline{B. $86\%$}\\
	\begin{itemize}
		\item $P(C=1)=0.12$
		\item $P(T=1|C=1)=0.90$
		\item $P(T=1|C=0)=0.02$
	\end{itemize}
	\begin{equation}
	\nonumber
	\begin{array}{rcl}
		P(C=1|T=1) & = & \frac{P(T=1|C=1)P(C=1)}{P(T=1|C=1)P(C=1)+P(T=1|C=0)P(C=0)} \\
				   & = & \frac{0.90\times0.12}{0.90\times0.12+0.02\times0.88} \\
				   & = & 0.86
	\end{array}
	\end{equation}
	
	\item What is the most suitable error function for gradient \textbf{descent} using logistic regression?
	\begin{itemize}
		\item [A.] The negative log-likelihood function
		\item [B.] The number of mistakes
		\item [C.] The squared error
		\item [D.] The log-likelihood function
	\end{itemize}
	\textbf{Answer:}
	\underline{A. The negative log-likelihood function}\\
	The negative log-likelihood function of a logistic regression is a convex function.
\end{enumerate}


\newpage

\section*{Part B, Problem 1: Bias-Variance Decomposition}

Consider a $p$-dimensional vector $\vec{x}\in\mathbb{R}^p$ drawn from a Gaussian distribution with an identity covariance matrix $\Sigma=I_p$ and an unknown mean $\vec{\mu}$, i.e. $\vec{x}\sim\mathcal{N}(\vec{\mu},I_p)$. Our goal is to evaluate the effectiveness of an estimator $\hat{\vec{\mu}}=f(\vec{x})$ of the mean from only a single sample (i.e. $n=1$) by measuring its mean squared error $\mathbb{E}[||\hat{\vec{\mu}}-\vec{\mu}||^2]$, where $||\cdot||^2$ is the squared Euclidean norm and the expectation is taken over the data generating distribution.

Note that for any estimator $\hat{\vec{\theta}}$ of a parameter vector $\vec{\theta}$, its mean squared error can be decomposed as:
$$\mathbb{E}[||\hat{\vec{\theta}}-\vec{\theta}||^2] = ||Bias[\hat{\vec{\theta}}]||^2 + trace(Var[\hat{\vec{\theta}}])$$
where,
$$Bias[\hat{\vec{\theta}}] = \mathbb{E}[\hat{\vec{\theta}}] - \vec{\theta}~and~Var[\hat{\vec{\theta}}]_{i,i}=Var[\hat{\theta}_i]=\mathbb{E}[(\hat{\theta}_i-\mathbb{E}[\hat{\theta}_j])^2]$$

\begin{enumerate}
	\item Derive the maximum likelihood estimator:
	$$\hat{\vec{\mu}}_{MLE}=\arg\max_{\vec{\mu}} P(\vec{x}|\vec{\mu})$$
	\\\textbf{Answer:}\\
	$\because$ We estimate the mean $\vec{\mu}$ from only a single sample $\vec{x}_1 \sim \mathcal{N}(\vec{\mu},I_p)$ \\
	$\therefore$ The likelihood function is $$L(\vec{\mu})=P(\vec{x}_1|\vec{\mu})=\frac{1}{(2\pi)^{p/2}|I_p|^{1/2}}\exp(-\frac{1}{2}(\vec{x}_1-\vec{\mu})^TI_p^{-1}(\vec{x}_1-\vec{\mu}))$$
	$\therefore$
	$$\hat{\vec{\mu}}_{MLE}=\arg\max_{\vec{\mu}}L(\vec{\mu})=\vec{x}_1$$
	
	What is its mean squared error?
	\\\textbf{Answer:}\\
	$\because$ The MSE of $\hat{\vec{\mu}}_{MLE}$ is
	\begin{equation}
	\nonumber
	\begin{array}{rcl}
		\mathbb{E}[||\hat{\vec{\mu}}_{MLE}-\vec{\mu}||^2] & = & ||Bias[\hat{\vec{\mu}}_{MLE}]||^2 + trace(Var[\hat{\vec{\mu}}_{MLE}]) \\
		& = & ||\mathbb{E}[\hat{\vec{\mu}}_{MLE}]-\vec{\mu}||^2 + \sum_{i=1}^{p}{\mathbb{E}[(\hat{\mu}_{MLE_i}-\mathbb{E}[\hat{\mu}_{MLE_i}])^2]}
	\end{array}
	\end{equation}
	$\because$ $\hat{\vec{\mu}}_{MLE}=\vec{x}_1$\\
	$\therefore$ $\mathbb{E}[\hat{\vec{\mu}}_{MLE}]=\mathbb{E}[\vec{x}_1]=\vec{\mu}$\\
	$\therefore$ $$\mathbb{E}[||\hat{\vec{\mu}}_{MLE}-\vec{\mu}||^2]=\sum_{i=1}^{p}{\mathbb{E}[(\vec{x}_{1_i}-\vec{\mu}_i)^2]}$$
	$\because$ $\vec{x}\sim\mathcal{N}(\vec{\mu},I_p)$\\
	$\therefore$ $$MSE=\mathbb{E}[||\hat{\vec{\mu}}_{MLE}-\vec{\mu}||^2]=p$$
	
	\item Derive the $\ell_2$-regularized maximum likelihood estimator:
	$$\hat{\vec{\mu}}_{RMLE}=\arg\max_{\vec{\mu}}\log P(\vec{x}|\vec{\mu})-\lambda ||\vec{\mu}||^2$$ 
	\\\textbf{Answer:}\\
	$\because$ We estimate the mean $\vec{\mu}$ from only a single sample $\vec{x}_1 \sim \mathcal{N}(\vec{\mu},I_p)$ \\
	$\therefore$ The $\ell_2$-regularized log-likelihood function is
	$$L(\vec{\mu}) = C - \frac{1}{2}(\vec{x}_1-\vec{\mu})^T(\vec{x}_1-\vec{\mu})-\lambda\vec{\mu}^T\vec{\mu}=C-\frac{1}{2}\vec{x}_1^T\vec{x}_1+\vec{x}_1^T\vec{\mu}-(\frac{1}{2}+\lambda)\vec{\mu}^T\vec{\mu}$$
	$\therefore$
	\begin{equation}
	\nonumber
	\begin{array}{rcl}
	\left.\frac{\partial L(\vec{\mu})}{\partial \vec{\mu}}\right|_{\vec{\mu}_{RMLE}} & = & \vec{0} \\
	\left.(\vec{x}_1-(1+2\lambda)\vec{\mu})\right|_{\vec{\mu}_{RMLE}} & = & \vec{0} \\
	\vec{\mu}_{RMLE} & = & \frac{1}{1+2\lambda}\vec{x}_1
	\end{array}
	\end{equation}
	
	What is its mean squared error?
	\\\textbf{Answer:}\\
	From question 1, we know that
	$$\mathbb{E}[||\hat{\vec{\mu}}_{RMLE}-\vec{\mu}||^2]=||\mathbb{E}[\hat{\vec{\mu}}_{RMLE}]-\vec{\mu}||^2 + \sum_{i=1}^{p}{\mathbb{E}[(\hat{\mu}_{RMLE_i}-\mathbb{E}[\hat{\mu}_{RMLE_i}])^2]}$$
	$\because$ $\vec{\mu}_{R MLE}=\frac{1}{1+2\lambda}\vec{x}_1$\\
	$\therefore$ $\mathbb{E}[\hat{\vec{\mu}}_{RMLE}]=\mathbb{E}[\frac{1}{1+2\lambda}\vec{x}_1]=\frac{1}{1+2\lambda}\vec{\mu}$\\
	$\therefore$ $$\mathbb{E}[||\hat{\vec{\mu}}_{RMLE}-\vec{\mu}||^2]=(\frac{2\lambda}{1+2\lambda})^2||\vec{\mu}||^2+(\frac{1}{1+2\lambda})^2\sum_{i=1}^{p}{\mathbb{E}[(\vec{x}_{1_i}-\vec{\mu}_i)^2]}$$
	$\because$ $\vec{x}\sim\mathcal{N}(\vec{\mu},I_p)$\\
	$\therefore$ $$MSE=\mathbb{E}[||\hat{\vec{\mu}}_{RMLE}-\vec{\mu}||^2]=(\frac{2\lambda}{1+2\lambda})^2||\vec{\mu}||^2+(\frac{1}{1+2\lambda})^2p$$
		
	\item Consider an estimator of the form $\hat{\vec{\mu}}_{SCALE} = c\vec{x}$ where $c\in \mathbb{R}$ is a constant scaling factor. Find the value $c^*$ that minimizes its mean squared error:
	$$c^* = \arg\min_c{\mathbb{E}[||c\vec{x}-\vec{\mu}||^2]}$$
	\\\textbf{Answer:}\\
	From question 2, if we assume $c=\frac{1}{1+2\lambda}$, we can easily get the objective function for $\hat{\vec{\mu}}_{SCALE} = c\vec{x}$:
	$$J_{MSE}(c)=(c-1)^2||\vec{\mu}||^2+c^2p$$
	$\therefore$
	\begin{equation}
	\nonumber
	\begin{array}{rcl}
	\left.\frac{dJ_{MSE}(c)}{dc}\right|_{c^*} & = & 0 \\
	\left.((||\vec{\mu}||^2+p)c-||\vec{\mu}||^2)\right|_{c^*} & = & 0 \\
	c^* & = & \frac{||\vec{\mu}||^2}{||\vec{\mu}||^2+p}
	\end{array}
	\end{equation}
	
	What is the corresponding minimum mean squared error?
	\\\textbf{Answer:}\\
	If $c^* = \frac{||\vec{\mu}||^2}{||\vec{\mu}||^2+p}$, then 
	$$MSE^*=J_{MSE}(c^*)=(\frac{p}{||\vec{\mu}||^2+p})^2||\vec{\mu}||^2+(\frac{||\vec{\mu}||^2}{||\vec{\mu}||^2+p})^2p=\frac{||\vec{\mu}||^2p}{||\vec{\mu}||^2+p}$$
	
	\item Consider the James-Stein estimator:
	$$\hat{\vec{\mu}}_{JS}=\left(1-\frac{p-2}{||\vec{x}||^2}\right)\vec{x}$$
	Note that $\hat{\vec{\mu}}_{JS}$ can be written as $\vec{x}-g(\vec{x})$ where $g(\vec{x})=\frac{p-2}{||\vec{x}||^2}\vec{x}$. This allows us to separate the mean squared error into three parts:
	\begin{equation}
	\nonumber
	\begin{array}{rcl}
	\mathbb{E}[||\hat{\vec{\mu}}_{JS}-\vec{\mu}||^2] & = & \mathbb{E}[||\vec{x}-g(\vec{x})-\vec{\mu}||^2] \\
												& = & \mathbb{E}[\vec{x}^T\vec{x}-2\vec{x}^T\vec{\mu}+\vec{\mu}^T\vec{\mu}+g(\vec{x}^Tg(\vec{x})-2\vec{x}^Tg(\vec{x})+2\vec{\mu}^Tg(\vec{x})] \\
												& = & \mathbb{E}[||\vec{x}-\vec{\mu}||^2]+\mathbb{E}[||g(\vec{x})||^2]-2\mathbb{E}[(\vec{x}-\vec{\mu})^Tg(\vec{x})]
	\end{array}
	\end{equation}
	Furthermore, from Stein's lemma, we know that: 
	$$\mathbb{E}[(\vec{x}-\vec{\mu})^Tg(\vec{x})] = \mathbb{E}\left[\sum_{j=1}^{p}{\frac{\partial}{\partial x_j}g_j(\vec{x})}\right]$$
	\begin{itemize}
		\item Find $\mathbb{E}[||\vec{x}-\vec{\mu}||^2]$.
		\\\textbf{Answer:}\\
		\begin{equation}
		\nonumber
		\begin{array}{rcl}
		\mathbb{E}[||\vec{x}-\vec{\mu}||^2] & = & \mathbb{E}[\sum_{i=1}^{p}{(x_i-\mu_i)^2}] \\
											& = & \sum_{i=1}^{p}{\mathbb{E}[(x_i-\mu_i)^2]} \\
		\end{array}
		\end{equation}
		$\because$ $\vec{x}\sim\mathcal{N}(\vec{\mu},I_p)$\\
		$\therefore$
		$$\mathbb{E}[||\vec{x}-\vec{\mu}||^2] = p$$
		
		\item Find $\mathbb{E}[||g(\vec{x})||^2]$. (Hint: your answer will include $\mathbb{E}[||\vec{x}||^{-2}]$)
		\\\textbf{Answer:}\\
		$\because$ $g(\vec{x})=\frac{p-2}{||\vec{x}||^2}\vec{x}$\\
		$\therefore$
		$$\mathbb{E}[||g(\vec{x})||^2] = (p-2)^2\mathbb{E}[\frac{\vec{x}^T\vec{x}}{(||\vec{x}||^2)^2}]=(p-2)^2\mathbb{E}[||\vec{x}||^{-2}]$$
		
		\item Show that:
		$$\frac{\partial}{\partial x_j}g_j(\vec{x}) = (p-2)\frac{||\vec{x}||^2-2x_j^2}{||\vec{x}||^4}$$
		where $x_j$ is the $j$th element of $x$ and $g_j(\vec{x})$ is the $j$th element of $g(\vec{x})$.
		\\\textbf{Answer:}\\
		$\because$ $g_j(\vec{x})=\frac{p-2}{||\vec{x}||^2}x_j$ \\
		$\therefore$ 
		\begin{equation}
		\nonumber
		\begin{array}{rcl}
		\frac{\partial}{\partial x_j}g_j(\vec{x}) & = & \frac{\partial}{\partial x_j}(\frac{p-2}{||\vec{x}||^2}x_j) \\
		& = & (p-2) (\frac{\partial (\sum_{i=1}^{p}{x^2_i})^{-1}}{\partial x_j}x_j+\frac{1}{||\vec{x}||^2}) \\
		& = & (p-2) (\frac{-2x_j^2}{(\sum_{i=1}^{p}{x^2_i})^{-2}}+\frac{1}{||\vec{x}||^2}) \\
		& = & (p-2)\frac{||\vec{x}||^2-2x_j^2}{||\vec{x}||^4}
		\end{array}
		\end{equation}
		
		\item What is the resulting mean squred error. (Hint: your answer will include $\mathbb{E}[||\vec{x}||^{-2}]$)
		\\\textbf{Answer:}\\
		$\because$
		\begin{itemize}
			\item $\mathbb{E}[||\vec{x}-\vec{\mu}||^2] = p$
			\item $\mathbb{E}[||g(\vec{x})||^2] = (p-2)^2\mathbb{E}[||\vec{x}||^{-2}]$
			\item $\frac{\partial}{\partial x_j}g_j(\vec{x}) = (p-2)\frac{||\vec{x}||^2-2x_j^2}{||\vec{x}||^4}$
		\end{itemize}
		$\therefore$\\
		\begin{equation}
		\nonumber
		\begin{array}{rcl}
		\mathbb{E}[||\hat{\vec{\mu}}_{JS}-\vec{\mu}||^2] & = & \mathbb{E}[||\vec{x}-\vec{\mu}||^2]+\mathbb{E}[||g(\vec{x})||^2]-2\mathbb{E}\left[\sum_{j=1}^{p}{\frac{\partial}{\partial x_j}g_j(\vec{x})}\right] \\
		& = & p + (p-2)^2\mathbb{E}[||\vec{x}||^{-2}] - 2(p-2)\mathbb{E}[\frac{p||\vec{x}||^2-2\sum_{j=1}^{p}{x_j^2}}{||\vec{x}||^4}] \\
		& = & p + (p-2)^2\mathbb{E}[||\vec{x}||^{-2}] - 2(p-2)^2\mathbb{E}[||\vec{x}||^{-2}] \\
		& = & p - (p-2)^2\mathbb{E}[||\vec{x}||^{-2}] \\
		\end{array}
		\end{equation}
		
	\end{itemize}

	\item Qualitatively compare these estimators, noting any similarities between them. How does regularization affect an estimator's bias and variance? Which estimator would you choose to approximate $\vec{\mu}$ from real data about which you have no prior knowledge? How does the data dimensionality $p$ affect your answer?
	\\\textbf{Answer:}\\
	\begin{itemize}
		\item Similarities:
		\begin{itemize}
			\item For $\hat{\vec{\mu}}_{RMLE}$, if $\lambda=0$, $\hat{\vec{\mu}}_{RMLE}=\hat{\vec{\mu}}_{MLE}=\vec{x}_1$
			\item For $\hat{\vec{\mu}}_{SCALE}$, if $c=1$, $\hat{\vec{\mu}}_{SCALE}=\hat{\vec{\mu}}_{MLE}=\vec{x}_1$
			\item For $\hat{\vec{\mu}}_{JS}$, if $p=2$, $\hat{\vec{\mu}}_{JS}=\hat{\vec{\mu}}_{MLE}=\vec{x}_1$
		\end{itemize}
		\item The regularization increases the bias, but decreases the variance.
		\item If I have no prior knowledge, I will choose MLE, because its MSE is not related with the prior knowledge of $\vec{\mu}$ and thus is predictable.
		\item The increase of dimensionality $p$ will increase the MSE except for the James-Stein estimator.\\
		$\because$ $\mathbb{E}[||\vec{x}||^{-2}] \geq 0$\\
		$\therefore$ $MSE(p)=-\mathbb{E}[||\vec{x}||^{-2}]p^2+(4\mathbb{E}[||\vec{x}||^{-2}]+1)p-4\mathbb{E}[||\vec{x}||^{-2}]$ must have maximum value.\\
		$\therefore$ If the dimensionality $p$ is very large, we can choose James-Stein estimator to constrain its MSE level.
	\end{itemize}
	
\end{enumerate}

\newpage

\section*{Part B, Problem 2: Linear Regression}
Suppose we observe $N$ data pairs $\{(x_i,y_i)\}_{i=1}^N$, where $y_i$ is generated by the following rule:
$$y_i=\vec{x}_i^T \vec{\beta} + \epsilon_i$$
where $\vec{x}_i,\vec{\beta}\in \mathbb{R}^d$, and $\epsilon_i$ is an i.i.d random noise drawn from the Gaussian Distribution:
$$\epsilon_i \sim \mathcal{N}(0,\sigma^2)$$
with a known constant $\sigma$. We further denote $\vec{Y}=[y_1,y_2,\dots,y_N]^T$ and $X=[\vec{x}_1,\vec{x}_2,...,\vec{x}_N]^T$.

Now, we are interested in estimating $\vec{\beta}$ from the observed data.
\begin{enumerate}
	\item Derive the likelihood function $\mathcal{L}(\vec{\beta})$
	\\\textbf{Answer:}\\
	$\because$ $y_i=\vec{x}_i^T \vec{\beta} + \epsilon_i$ and $\epsilon_i \sim \mathcal{N}(0,\sigma^2)$\\
	$\therefore$ $y_i \sim \mathcal{N}(\vec{x}_i^T\vec{\beta},\sigma^2)$\\
	$\therefore$ 
	\begin{equation}
	\nonumber
	\begin{array}{rcl}
	\mathcal{L}(\vec{\beta}) & = & \prod_{i=1}^{N}P(y_i|\vec{x}_i,\vec{\beta}) \\
							 & = & \prod_{i=1}^{N}(\frac{1}{\sqrt{2\pi}\sigma}\exp(-\frac{(y_i-\vec{x}_i^T\vec{\beta})^2}{2\sigma^2}))
	\end{array}
	\end{equation}
	
	\item Show that the MLE estimator $\hat{\vec{\beta}}_{MLE}$ of $\vec{\beta}$ is equivalent to the solution of the following linear regression problem:
	\begin{equation}
	\label{eq:MLE}
	\min_{\vec{\beta}}\frac{1}{2}||\vec{Y}-X\vec{\beta}||_2^2
	\end{equation}
	\\\textbf{Answer:}\\
	$\because$ We can derive MLE estimator $\hat{\vec{\beta}}_{MLE}$ via the log likelihood:
	\begin{equation}
	\nonumber
	\begin{array}{rcl}
	\hat{\vec{\beta}}_{MLE} & = & \arg\max_{\vec{\beta}} \log{\mathcal{L}(\vec{\beta})} \\
							& = & \arg\max_{\vec{\beta}} \sum_{i=1}^{N}{-\frac{1}{2\sigma^2}(y_i-\vec{x}^T\vec{\beta})^2} \\
							& = & \arg\min_{\vec{\beta}} \frac{1}{2}\sum_{i=1}^{N}{(y_i-\vec{x}^T\vec{\beta})^2} \\
							& = & \arg\min_{\vec{\beta}} \frac{1}{2}||\vec{Y}-X\vec{\beta}||_2^2
	\end{array}
	\end{equation}
	$\therefore$ The MLE estimator is equivalent to the solution of the following linear regression problem: $$J^*(\vec{\beta})=\min_{\vec{\beta}}\frac{1}{2}||\vec{Y}-X\vec{\beta}||_2^2$$
	
	\item Now we suppose $\vec{\beta}$ is not a deterministic parameter, but a random variable having a Gaussian prior distribution:
	$$p(\vec{\beta}) \sim \mathcal{N}(\vec{0},\frac{\sigma^2}{2\lambda}I_d)$$
	where $I$ is a $d \times d$ identity matrix and $\lambda>0$ is a known parameter. Show that the MAP estimation $\hat{\vec{\beta}}_{MAP}$ of $\vec{\beta}$ is equivalent to the solution of the following ridge regression problem:
	\begin{equation}
	\label{eq:MAP}
	\min_{\vec{\beta}} \frac{1}{2} ||\vec{Y}-X\vec{\beta}||_2^2+\lambda||\vec{\beta}||_2^2
	\end{equation}
	\\\textbf{Answer:}\\
	$\because$ $p(\vec{\beta}) \sim \mathcal{N}(\vec{0},\frac{\sigma^2}{2\lambda}I_d)$\\
	$\therefore$
	$$P(\vec{\beta}) = \frac{1}{(2\pi)^{d/2}|\frac{\sigma^2}{2\lambda}I_d|^{1/2}}\exp(-\frac{1}{2}\vec{\beta}^T(\frac{\sigma^2}{2\lambda}I_d)^{-1}\vec{\beta})=\frac{1}{(\frac{\pi}{\lambda})^{d/2}\sigma^d}\exp(-\frac{\lambda}{\sigma^2}\vec{\beta}^T\vec{\beta})$$
	$\because$ the posteriori distribution $P(\vec{\beta}|y_i,\vec{x}_i) \propto P(y_i,|\vec{x}_i,\vec{\beta})P(\vec{\beta})$ \\
	$\therefore$ the MAP estimator $\hat{\vec{\beta}}_{MAP}$ can be derived from
	\begin{equation}
	\nonumber
	\begin{array}{rcl}
	\hat{\vec{\beta}}_{MAP} & = & \arg\max_{\vec{\beta}}{\prod_{i=1}^{N}{P(y_i,|\vec{x}_i,\vec{\beta})P(\vec{\beta})}} \\
							& = & \arg\max_{\vec{\beta}}\sum_{i=1}^{N}(-\frac{1}{2\sigma^2}(y_i-\vec{x}^T\vec{\beta})^2)-\frac{\lambda}{\sigma^2}\vec{\beta}^T\vec{\beta} \\
							& = & \arg\min_{\vec{\beta}}\frac{1}{2}\sum_{i=1}^{N}(y_i-\vec{x}^T\vec{\beta})^2+\lambda\vec{\beta}^T\vec{\beta} \\
							& = & \arg\min_{\vec{\beta}}\frac{1}{2}||\vec{Y}-X\vec{\beta}||_2^2 + \lambda||\vec{\beta}||_2^2\\
	\end{array}
	\end{equation}
	$\therefore$ The MAP estimator is equivalent to the solution of the following ridge regression problem:
	$$J^*(\vec{\beta})=\min_{\vec{\beta}} \frac{1}{2} ||\vec{Y}-X\vec{\beta}||_2^2+\lambda||\vec{\beta}||_2^2$$
	
	\item Refer to the closed form solutions of (\ref{eq:MLE}) and (\ref{eq:MAP}) in the lecture slides, what might be an issue of $\hat{\vec{\beta}}_{MLE}$ if $d>>N$? How can $\hat{\vec{\beta}}_{MAP}$ possibly address it?
	\\\textbf{Answer:}\\
	From the lecture, we know the closed form solutions of (\ref{eq:MLE}) and (\ref{eq:MAP}) as below:
	\begin{itemize}
		\item $\hat{\vec{\beta}}_{MLE} = (X^TX)^{-1}X^TY$
		\item $\hat{\vec{\beta}}_{MAP} = (X^TX+\lambda I)^{-1}X^TY$
	\end{itemize}
	$\because$ $X$ is a $N\times d$ matrix.\\
	$\therefore$ $X^TX$ is a $d \times d$ matrix. \\
	$\because$ $rank(X^TX) \leq rank(X) \leq min(d,N) = N << d$ \\
	$\therefore$ $X^TX$ is not invertible.\\
	$\therefore$ $\hat{\vec{\beta}}_{MLE}$ is not feasible. \\
	$\because$ $(X^TX+\lambda I)$ is a positive definite matrix, if $\lambda>0$ \\
	$\therefore$ $(X^TX+\lambda I)$ is invertible. \\
	$\therefore$ $\hat{\vec{\beta}}_{MAP}$ is feasible.
\end{enumerate}

\newpage

\section*{Part B, Problem 3: MLE, MAP and Logistic Regression}

We learnt about Maximum Likelihood estimation in class. For a fixed set of data and underlying statistical model, the method of maximum likelihood selects the set of values of the model parameters that maximizes the likelihood function.

In this problem, we will look at two different ways of estimating parameters in a probability distribution. Suppose we observe $n$ i.i.d. random variables $X_1,...,X_n$, drawn from a distribution with parameter $\theta$. That is, for each $X_i$ and a natural number $k$,
$$P(X_i=k)=(1-\theta)^k\theta$$
Given some observed values of $X_1$ to $X_n$ , we want to estimate the value of $\theta$.

\subsection*{3.1 Maximum Likelihood Estimation}

The first kind of estimator for $\theta$ we will consider is the Maximum Likelihood Estimator (MLE). The probability of observing given data is called the likelihood of the data, and the function that gives the likelihood for a given parameter $\hat{\theta}$ (which may or may not be equal to the true parameter $\theta$) is called the likelihood function, written as $L(\hat{\theta})$. When we use MLE, we estimate $\theta$ by choosing the $\hat{\theta}$ that maximizes the likelihood.
$$\hat{\theta}_{MLE}=\arg\max_{\hat{\theta}}L(\hat{\theta})$$

It is often convenient to deal with the log-likelihood ($\ell(\hat{\theta})=\log L(\hat{\theta})$) instead, and since log is an increasing function, the argmax also applies in the log space:
$$\hat{\theta}_{MLE}=\arg\max_{\hat{\theta}}\ell(\hat{\theta})$$

\begin{enumerate}
	\item Given a dataset $\mathcal{D}$, containing observations $\{X_1=k_1,X_2=k_2,\dots,X_n=k_n\}$, write an expression for $\ell(\hat{\theta})$ as a function of $\mathcal{D}$ and $\hat{\theta}$. How does the order of the variables affect the function?
	\\\textbf{Answer:}\\
	$\because$ $P(X_i=k_i|\theta)=(1-\theta)^{k_i}\theta$\\
	$\therefore$
	\begin{equation}
	\nonumber
	\begin{array}{rcl}
	\ell(\hat{\theta}) & = & \prod_{i=1}^{n}P(X_i=k_i|\hat{\theta}) \\
					   & = & (1-\theta)^{\sum_{i=1}^{n}{k_i}}\theta^n \\
	\end{array}
	\end{equation}
	
	\item Derive an expression for the maximum likelihood estimate.
	\\\textbf{Answer:}\\
	Assume $K=\sum_{i=1}^{n}{k_i}$
	\begin{equation}
	\nonumber
	\begin{array}{rcl}
	\hat{\theta}_{MLE} & = & \arg\max_{\hat{\theta}}\ell(\hat{\theta}) \\
	\left. \frac{d\ell(\hat{\theta})}{d\hat{\theta}}\right|_{\hat{\theta}_{MLE}}	& = & 0 \\
	\left. ((1-\theta)^{K-1}\theta^{n-1}(n-(n+K)\theta))\right|_{\hat{\theta}_{MLE}} & = & 0 \\
	\hat{\theta}_{MLE} & = & \frac{n}{n+K} = \frac{n}{n+\sum_{i=1}^{n}{k_i}} \\
	\end{array}
	\end{equation}
\end{enumerate}

\subsection*{3.2 Maximum a Posteriori Estimation}

Now we assume that we have some prior knowledge about the true parameter $\theta$. We express it by treating $\theta$ itself as a random variable and defining a prior probability distribution over it. Precisely, we suppose that the data $X_1,\dots,X_n$ are drawn as follows:
\begin{itemize}
	\item $\theta$ is drawn from the prior probability distribution
	\item Then $X_1,\dots,X_n$ are drawn independently from a Geometric distribution with $\theta$ as the parameter.
\end{itemize}
Now both $X_i$ and $\theta$ are random variables, and they have a joint probability distribution. We now estimate $\theta$ as follows
$$\hat{\theta}_{MAP}=\arg\max_{\hat{\theta}}P(\theta=\hat{\theta}|X_1,\dots,X_n)$$

This is called Maximum a Posteriori (MAP) estimation. Using Bayes rule, we can rewrite the posterior probability as follows.
$$P(\theta=\hat{\theta}|X_i,\dots,X_n)=\frac{P(X_i,\dots,X_n|\theta=\hat{\theta})P(\theta=\hat{\theta})}{P(X_1,\dots,X_n)}$$

Applying this to the MAP estimate, we get the following expression. Notice that we can ignore the denominator since it is not a function of $\hat{\theta}$
\begin{equation*}
\begin{array}{rcl}
\hat{\theta}_{MAP} & = & \arg\max_{\hat{\theta}}P(X_1,\dots,X_n|\theta=\hat{\theta})P(\theta=\hat{\theta}) \\
				   & = & \arg\max_{\hat{\theta}}L(\hat{\theta})P(\theta=\hat{\theta}) \\
				   & = & \arg\max_{\hat{\theta}}(\ell(\hat{\theta})+\log P(\theta=\hat{\theta})) \\
\end{array}
\end{equation*}

Thus, the MAP estimator maximizes the sum of the log-likelihood and the log-probability of the prior distribution on $\theta$. WHen the prior is a continuous distribution with density function $p$, we have
$$\hat{\theta}_{MAP}=\arg\max_{\hat{\theta}}(\ell(\hat{\theta})+\log{p(\hat{\theta})})$$

For this problem, we will use the Beta distribution (a popular choice when the data dstribution is Geometric or Bernoulli) as the prior, and the density function is given by
$$p(\hat{\theta})=\frac{\hat{\theta}^{\alpha-1}(1-\hat{\theta})^{\beta-1}}{B(\alpha,\beta)}$$
where $B(\alpha,\beta)$ is the beta function.

\begin{enumerate}
	\setcounter{enumi}{3}
	\item Derive a close form expression for the maximum a posteriori estimate. (hint: If $x^*$ maximizes $f$, $f'(x^*)=0$).
	\\\textbf{Answer:}\\
	$\because$ $P(X_i=k_i|\theta)=(1-\theta)^{k_i}\theta$ and $p(\hat{\theta})=\frac{\hat{\theta}^{\alpha-1}(1-\hat{\theta})^{\beta-1}}{B(\alpha,\beta)}$ \\
	$\therefore$ The MAP estimate of $\hat{\theta}$ can be derived by (assume $K=\sum_{i=1}^{n}k_i$):
	\begin{equation}
	\nonumber
	\begin{array}{rcl}
	\hat{\theta}_{MAP} & = & \arg\max_{\hat{\theta}}\prod_{i=1}^{n}{p(X_i=k_i|\hat{\theta})p(\hat{\theta})} \\
					   & = & \arg\max_{\hat{\theta}}(1-\hat{\theta})^K\hat{\theta}^n p(\hat{\theta}) \\
					   & = & \arg\max_{\hat{\theta}}K\ln{(1-\hat{\theta})}+n\ln{\hat{\theta}}+\ln{p(\hat{\theta})} \\
					   & = & \arg\max_{\hat{\theta}}K\ln{(1-\hat{\theta})}+n\ln{\hat{\theta}}+(\alpha-1)\ln{\hat{\theta}}+(\beta-1)\ln{(1-\hat{\theta})} \\
					   & = & \arg\max_{\hat{\theta}}(n+\alpha-1)\ln{\hat{\theta}}+(K+\beta-1)\ln{(1-\hat{\theta})} \\
	\end{array}
	\end{equation}
	Assume the objective function $J(\hat{\theta})=(n+\alpha-1)\ln{\hat{\theta}}+(K+\beta-1)\ln{(1-\hat{\theta})}$, then
	\begin{equation}
	\nonumber
	\begin{array}{rcl}
	\left. \frac{dJ(\hat{\theta})}{d\hat{\theta}}\right|_{\hat{\theta}_{MAP}} & = & 0 \\
	\left. (\frac{n+\alpha-1}{\hat{\theta}}-\frac{K+\beta-1}{1-\hat{\theta}})\right|_{\hat{\theta}_{MAP}} & = & 0 \\
	\hat{\theta}_{MAP} & = & \frac{n+\alpha-1}{K+n+\alpha+\beta-2}
	\end{array}
	\end{equation}
	
	\item Is the bias of Maximum Likelihood Estimate (MLE) typically greater than or equal to the bias of Maximum A Posteriori (MAP) estimate? (Explain your answer in a sentence)
	\\\textbf{Answer:}\\
	No, it depends on the difference between the prior distribution and real distribution of parameters.
	
	\item What can you say about the value of Maximum Likelihood Estimate (MLE) as compared to the value of Maximum A Posteriori (MAP) estimate with a uniform prior? Why?
	\\\textbf{Answer:}\\
	If the prior is a uniform distribution, then with such a weak priori, the MLE is same with MAP. Because the priori has no effect on the MAP estimation, and only the shared likelihood between both estimators will determine the estimation value.
\end{enumerate}

\subsection*{3.3 Logistic Regression}

In class, we wil learn about MLE of parametrers in logistic regression. For a given data $\vec{x}\in \mathbb{R}^p$, the probability of $Y$ being 1 in logistic regression is
\begin{equation}
\label{eq:logistic}
P(Y=1|\vec{X}=\vec{x})=\frac{\exp(w_0+\vec{x}^T\vec{w})}{1+\exp(w_0+\vec{x}^T\vec{w})}
\end{equation}
where $w_0$ and $\vec{w}=(w_1,w_2,\dots,w_p)^T$ are model parameters. In this problem, we consider the maximum a posteriori setting, where we put a Gaussian prior on the parameters:
$$w_i\sim \mathcal{N}(\mu,1),~for~i=0,1,2,\dots,p.$$

\begin{enumerate}
	\setcounter{enumi}{6}
	\item Choose a conjugate prior for Gaussian on $\mu$ (choose any higher parameters as you want to ease the computation). Assuming you are given a dataset with $n$ training examples and $p$ features, write down a formula for the conditional log posterior likelihood of the training data in terms of the class labels $y^{(i)}$, the features $x_1^{(i)},\dots,x_p^{(i)}$, and the parameters $w_0,w_1,\dots,w_p$, where the superscript $(i)$ denotes the sample index. This will be your objective function for gradient ascent.
	\\\textbf{Answer:}\\
	Choose $\mu \sim \mathcal{N}(\tilde{\mu},\sigma^2)$ as a conjugate prior for Gaussian on $\mu$, then the conditional log posterior likelihood of the training data is (assume $\vec{\mathcal{Y}}=[y^{(1)},y^{(2)},\dots,y^{(n)}]$, and $\mathcal{X}=[\vec{x}^{(1)},\vec{x}^{(2)},\dots,\vec{x}^{(n)}]^T$):
	\begin{equation}
	\nonumber
	\begin{array}{rcl}
	f(w_0, \vec{w},\mu) & = & \ln(P(\vec{\mathcal{Y}}|\mathcal{X},w_0,\vec{w},\mu)P(w_0,\vec{w}|\mu)P(\mu)) \\
						& = & \sum_{i=1}^{n}\ln(P(y^{(i)}|\vec{x}^{(i)},w_0,\vec{w},\mu))+\sum_{j=0}^{p}\ln(P(w_j|\mu))+\ln(P(\mu)) \\
						& = & \sum_{i=1}^{n}(y^{(i)}(w_0+\sum_{k=1}^{p}w_kx_k^{(i)})-\ln(1+\exp(w_0+\sum_{k=1}^{p}w_kx_k^{(i)})))\\
						&   & -\frac{1}{2}\sum_{j=0}^{p}{(w_j-\mu)^2} - \frac{1}{2\sigma^2}(\mu-\tilde{\mu})^2 + C
	\end{array}
	\end{equation}
	
	\item Compute the partial derivative of the objective with respect to $w_0$, to an arbitrary $w_i$, and $\mu$, i.e. derive $\partial f/\partial w_0$, $\partial f/\partial w_i$, $\partial f/\partial \mu$ where $f$ is the objective that you provided above. Use (\ref{eq:logistic}) to simplify the formula. What is the MAP estimation of $\mu$ given $w_0$ and $\vec{w}$
	\\\textbf{Answer:}\\
	\begin{itemize}
		\item $\partial f/\partial w_0$:
		\begin{equation}
		\nonumber
		\begin{array}{rcl}
		\partial f/\partial w_0 & = & \sum_{i=1}^{n}(y^{(i)}-\frac{\exp(w_0+\sum_{k=1}^{p}w_kx_k^{(i)})}{1+\exp(w_0+\sum_{k=1}^{p}w_kx_k^{(i)})}) \\
								& = & \sum_{i=1}^{n}(y^{(i)}-P(Y^{(i)}=1|\vec{x}^{(i)},w_0,\vec{w})) \\
		\end{array}
		\end{equation}
		\item $\partial f/\partial w_i$:
		\begin{equation}
		\nonumber
		\begin{array}{rcl}
		\partial f/\partial w_i & = & \sum_{j=1}^{n}(y^{(j)}x_i^{(j)}-\frac{x_i^{(j)}\exp(w_0+\sum_{k=1}^{p}w_kx_k^{(i)})}{1+\exp(w_0+\sum_{k=1}^{p}w_kx_k^{(j)})}) \\
		& = & \sum_{j=1}^{n}x_i^{(j)}(y^{(j)}-P(Y^{(j)}=1|\vec{x}^{(j)},w_0,\vec{w}))
		\end{array}
		\end{equation}
		\item $\partial f/\partial \mu$:
		\begin{equation}
		\nonumber
		\begin{array}{rcl}
		\partial f/\partial \mu & = & \sum_{j=0}^{p}(w_j-\mu) - \frac{1}{\sigma^2}(\mu-\tilde{\mu}) \\
		\left. \partial f/\partial \mu \right|_{\hat{\mu}_{MAP}} & = & 0 \\
		\left. \sum_{j=0}^{p}(w_j-\mu) - \frac{1}{\sigma^2}(\mu-\tilde{\mu}) \right|_{\hat{\mu}_{MAP}} & = & 0 \\
		\hat{\mu}_{MAP} & = & \frac{\sum_{j=0}^{p}w_j + \tilde{\mu}/\sigma^2}{p+1+1/\sigma^2}
		\end{array}
		\end{equation}
	\end{itemize}
\end{enumerate}

\newpage

\section*{Part C: Programming Exercise}

\subsection*{Exploring The Effect of Priors in Batting Average Estimation}

In this problem, you will explore how prior knowledge can effect your estimates of batting averages.

\subsubsection*{Dataset}

In this problem, we have generated data for 5000 fictional baseball players. The data is divided into 3 parts -- `\textit{pre\_season.txt}', `\textit{mid\_season.txt}', and `\textit{end\_season.txt}'. Each of these files has 3 columns: the \textit{id} for the player (an integer), the number of \textit{at\_bats} for the player (an at-bat is an opportunity to get a hit), and the number of \textit{hits} the player got during those at-bats. The data files can be loaded using the provided \textit{load\_data} function in \textit{hw1\_baseball.py}. The batting average for a player can be computed by dividing the number of hits by the number of \textit{at\_bats}.

\subsubsection*{Maximum Likelihood Estimator}

Assume for the momen that you only have access to the data in `\textit{mid\_season.txt}'. Midway through the season, you would like to estimate the end of season batting averages for all 5000 players. Write a function to compute the maximum likelihood estimate of the batting average for all 5000 players. Make sure to turn in your code.

\subsubsection*{Maximum a Posteriori Estimator}

Unsatisfied with the MLE estimates, you decide that you would like to use the pre-season statistics of the players as a prior on what their in-season batting average will be. Write a function to compute the maximum a posteriori estimate of the batting average for all 5000 players. Briefly describe how you choose to incorporate prior information. Make sure to turn in your code.

\subsubsection*{Visualize Your Estimates}

Compute the actual batting averages from `\textit{end\_season.txt}' (do not include statistics from the other files in these actual averages) and compare your estimates of the batting average to these estimates. Use the provided visualize function in \textit{hw1\_baseball.py} to visualize and compare your MLE and MAP estimators. Make sure to turn in your visualizations.
\begin{itemize}
	\item Does the MLE estimator appear to fail in certain cases? Why?
	\item Does the MAP estimator appear to fail in certain cases? Why?
	\item What conclusions do you draw from this experiment?
\end{itemize}


\newpage


\end{document}
