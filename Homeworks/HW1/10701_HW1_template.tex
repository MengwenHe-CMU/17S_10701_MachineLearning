\documentclass{article}

\usepackage[usenames,dvipsnames]{color}
\usepackage{bm}
\usepackage{amsmath}
\usepackage{amssymb}
\usepackage{graphicx}
\usepackage[colorlinks=true,urlcolor=blue]{hyperref}
\usepackage{geometry}
\geometry{margin=1in}
\usepackage{float}
\setlength{\marginparwidth}{2.15cm}
\usepackage{booktabs}
\usepackage{epsfig}
\usepackage{setspace}
\usepackage{parskip}
\usepackage[]{algorithm2e}
\usepackage{comment}
\usepackage{pdfpages}
%\usepackage{physics}
\usepackage{enumerate}

%\newcommand{\comment}[1]{\textcolor{blue}{\textsc{\textbf{[#1]}}}}

 \newenvironment{soln}{
     \leavevmode\color{blue}\ignorespaces
 }{}

\makeatletter
\newcommand{\removelatexerror}{\let\@latex@error\@gobble}
\makeatother

\begin{document}

\section*{}
\begin{center}
  \centerline{\textsc{\LARGE Homework 1}}
  \vspace{0.5em}
  \centerline{\textsc{\Large MLE, MAP estimates; Linear and Logistic Regression}}
  \vspace{1em}
  \textsc{\large CMU 10-701: Machine Learning (Spring 2017)} \\
  \vspace{1em}
  \centerline{OUT: Jan 31}
  \centerline{DUE: Feb 10, 11:59 PM}
  \centerline{NAME: Mengwen He}
  \centerline{ADREW ID: mengwenh}
	
\end{center}

\section*{Part A: Multiple Choice Questions}

\begin{enumerate}
	\item For each case listed below, what type of machine learning problem does it belong to?
	\begin{enumerate}
		\item Advertisement selection system, which can predict the probability whether a customer will click on an ad or not based on the search history. \\
		\textbf{Answer:}
		\underline{B. Supervised learning: Regression}\\
		A task, with ads click statistics and search history as input data, outputs the prediction of continuous probability of clicking an ad.
		\item U.S post offices use a system to automatically recognize handwriting on the envelope. \\
		\textbf{Answer:}
		\underline{A. Supervised learning: Classification}\\
		A task, with handwriting samples and their labels as input data, outputs the prediction of discrete numbers/letters of a handwriting on the envelope.
		\item Reduce dimensionality using principal components analysis (PCA). \\
		\textbf{Answer:}
		\underline{C. Unsupervised learning}\\
		A task, without training data as input, outputs a description of reduced dimensionality. 
		\item Trading companies try to predict future stock market based on current market conditions. \\
		\textbf{Answer:}
		\begin{itemize}
			\item \underline{A. Supervised learning: Classification}\\
			A task, with current market conditions as input, outputs the prediction of discrete stock market conditions, say bull or bear market.
			\item \underline{B. Supervised learning: Regression}\\
			A task, with current market conditions as input, outputs the prediction of continuous stock market conditions, say stock price.
		\end{itemize}
		
		\item Repair a digital image that has been partially damaged. \\
		\textbf{Answer:}
		\begin{itemize}
			\item \underline{A. Supervised learning: Classification}\\
			A task, with digital images database as input, outputs the prediction of discrete pixel value in the damaged zone.
			\item \underline{B. Supervised learning: Regression}\\
			A task, with digital images database as input, outputs the prediction of continuous parameters of a color distribution model to form a discrete patch to cover the damaged zone.
			\item \underline{C. Unsupervised learning}\\
			A task, without training data as input, outputs a description of a damaged pixel according to its surrounding pixel values, e.g. interpolation or extrapolation.
		\end{itemize}
	\end{enumerate}
	Type of machine learning problem:
	\begin{itemize}
		\item [A.] Supervised learning: Classification
		\item [B.] Supervised learning: Regression
		\item [C.] Unsupervised learning
	\end{itemize}
	
	\item For four statements below, which one is wrong?
	\begin{itemize}
		\item [A.] In maximum a posterior (MAP) estimate, data overwhelms the prior if we have enough data.
		\item [B.] There are no parameters in non-parametirc models.
		\item [C.] $P(X \cap Y \cap Z)=P(Z|X \cap Y) P(Y|X) P(X)$.
		\item [D.] Compared with parametric models, non-parameter models are flexible, since they don't make strong assumptions.
	\end{itemize}
	\textbf{Answer:}
	\underline{B. There are no parameters in non-parametirc models.} is wrong.\\
	Non-parametric model still needs parameters to describe the model, but the number of model's parameters is not fixed and will grow with the data size. The non-parametric only means that there is weak assumption on the model's type defined by a fixed number of parameters.
	
	\item There are about $12\%$ people in U.S. having breast cancer during their lifetime. One patient has a positive result for the medical test. Suppose the sensitivity of this test is $90\%$, meaning the test will be positive with probability $0.9$ if one really has cancer. The false positive is likely to be $2\%$. Then what is the probability this patient actually having cancer based on Bayes Theorem?\\
	A. $90\%$ \hspace{0.1\textwidth} B. $86\%$ \hspace{0.1\textwidth} C. $12\%$ \hspace{0.1\textwidth} D. $43\%$\\
	\textbf{Answer:}
	\underline{B. $86\%$}\\
	\begin{itemize}
		\item $P(C=1)=0.12$
		\item $P(T=1|C=1)=0.90$
		\item $P(T=1|C=0)=0.02$
	\end{itemize}
	\begin{equation}
	\nonumber
	\begin{array}{rcl}
		P(C=1|T=1) & = & \frac{P(T=1|C=1)P(C=1)}{P(T=1|C=1)P(C=1)+P(T=1|C=0)P(C=0)} \\
				   & = & \frac{0.90\times0.12}{0.90\times0.12+0.02\times0.88} \\
				   & = & 0.86
	\end{array}
	\end{equation}
	
	\item What is the most suitable error function for gradient \textbf{descent} using logistic regression?
	\begin{itemize}
		\item [A.] The negative log-likelihood function
		\item [B.] The number of mistakes
		\item [C.] The squared error
		\item [D.] The log-likelihood function
	\end{itemize}
	\textbf{Answer:}
	\underline{A. The negative log-likelihood function}\\
	The negative log-likelihood function of a logistic regression is a convex function.
\end{enumerate}


\newpage

\section*{Part B, Problem 1: Bias-Variance Decomposition}

Consider a $p$-dimensional vector $\vec{x}\in\mathbb{R}^p$ drawn from a Gaussian distribution with an identity covariance matrix $\Sigma=I_p$ and an unknown mean $\vec{\mu}$, i.e. $\vec{x}\sim\mathcal{N}(\vec{\mu},I_p)$. Our goal is to evaluate the effectiveness of an estimator $\hat{\vec{\mu}}=f(\vec{x})$ of the mean from only a single sample (i.e. $n=1$) by measuring its mean squared error $\mathbb{E}[||\hat{\vec{\mu}}-\vec{\mu}||^2]$, where $||\cdot||^2$ is the squared Euclidean norm and the expectation is taken over the data generating distribution.

Note that for any estimator $\hat{\vec{\theta}}$ of a parameter vector $\vec{\theta}$, its mean squared error can be decomposed as:
$$\mathbb{E}[||\hat{\vec{\theta}}-\vec{\theta}||^2] = ||Bias[\hat{\vec{\theta}}]||^2 + trace(Var[\hat{\vec{\theta}}])$$
where,
$$Bias[\hat{\vec{\theta}}] = \mathbb{E}[\hat{\vec{\theta}}] - \vec{\theta}~and~Var[\hat{\vec{\theta}}]=\mathbb{E}[(\hat{\vec{\theta}}-\vec{\theta})(\hat{\vec{\theta}}-\vec{\theta})]$$

\begin{enumerate}
	\item Derive the maximum likelihood estimator:
	$$\hat{\vec{\mu}}_{MLE}=\arg\max_{\vec{\mu}} P(\vec{x}|\vec{\mu})$$
	\\\textbf{Answer:}
	
	What is its mean squared error?
	\\\textbf{Answer:}
	
	\item Derive the $\ell_2$-regularized maximum likelihood estimator:
	$$\hat{\vec{\mu}}_{MLE}=\arg\max_{\vec{\mu}}\log P(\vec{x}|\vec{\mu})-\lambda ||\vec{\mu}||^2$$ 
	\\\textbf{Answer:}
	
	What is its mean squared error?
	\\\textbf{Answer:}
	
	\item Consider an estimator of the form $\hat{\vec{\mu}}_{SCALE} = c\vec{x}$ where $c\in \mathbb{R}$ is a constant scaling factor. Find the value $c^*$ that minimizes its mean squared error:
	$$c^* = \arg\min_c{\mathbb{E}[||c\vec{x}-\vec{\mu}||^2]}$$
	\\\textbf{Answer:}
	
	What is the corresponding minimum mean squared error?
	\\\textbf{Answer:}
	
	\item Consider the James-Stein estimator:
	$$\hat{\vec{\mu}}_{JS}=\left(1-\frac{p-2}{||\vec{x}||^2}\right)\vec{x}$$
	Note that $\hat{\vec{\mu}}_{JS}$ can be written as $\vec{x}-g(\vec{x})$ where $g(\vec{x})=\frac{p-2}{||\vec{x}||^2}\vec{x}$. This allows us to separate the mean squared error into three parts:
	\begin{equation}
	\nonumber
	\begin{array}{rcl}
	\mathbb{E}[||\hat{\vec{\mu}}-\vec{\mu}||^2] & = & \mathbb{E}[||\vec{x}-g(\vec{x}-\vec{\mu})||^2] \\
												& = & \mathbb{E}[\vec{x}^T\vec{x}-2\vec{x}^T\vec{\mu}+\vec{\mu}^T\vec{\mu}+g(\vec{x}^Tg(\vec{x})-2\vec{x}^Tg(\vec{x})+2\vec{\mu}^Tg(\vec{x})] \\
												& = & \mathbb{E}[||\vec{x}-\vec{\mu}||^2]+\mathbb{E}[||g(\vec{x})||^2]-2\mathbb{E}[(\vec{x}-\vec{\mu})^Tg(\vec{x})]
	\end{array}
	\end{equation}
	Furthermore, from Stein's lemma, we know that: 
	$$\mathbb{E}[(\vec{x}-\vec{\mu})^Tg(\vec{x})] = \mathbb{E}\left[\sum_{j=1}^{p}{\frac{\partial}{\partial x_j}g_j(\vec{x})}\right]$$
	\begin{itemize}
		\item Find $\mathbb{E}[||\vec{x}-\vec{\mu}||^2]$.
		\\\textbf{Answer:}
		
		\item Find $\mathbb{E}[||g(\vec{x})||^2]$. (Hint: your answer will include $\mathbb{E}[||\vec{x}||^{-2}]$)
		\\\textbf{Answer:}
		
		\item Show that:
		$$\frac{\partial}{\partial x_j}g_j(\vec{x}) = \frac{||\vec{x}||^2-2x_j^2}{||\vec{x}||^4}$$
		where $x_j$ is the $j$th element of $x$ and $g_j(\vec{x})$ is the $j$th element of $g(\vec{x})$.
		\\\textbf{Answer:}
		
		\item What is the resulting mean squred error. (Hint: your answer will include $\mathbb{E}[||\vec{x}||^{-2}]$)
		\\\textbf{Answer:}
		
	\end{itemize}

	\item Qualitatively compare these estimators, noting any similarities between them. How does regularization affect an estimator's bias and variance? Which estimator would you choose to approximate $\vec{\mu}$ from real data about which you have no prior knowledge? How does the data dimensionality $p$ affect your answer?
	\\\textbf{Answer:}
	
\end{enumerate}

\newpage

\section*{Part B, Problem 2: Linear Regression}
Suppose we observe $N$ data pairs $\{(x_i,y_i)\}_{i=1}^N$, where $y_i$ is generated by the following rule:
$$y_i=\vec{x}_i^T \vec{\beta} + \epsilon_i$$
where $\vec{x}_i,\vec{\beta}\in \mathbb{R}^d$, and $\epsilon_i$ is an i.i.d random noise drawn from the Gaussian Distribution:
$$\epsilon_i \sim \mathcal{N}(0,\sigma^2)$$
with a known constant $\sigma$. We further denote $\vec{Y}=[y_1,y_2,\dots,y_N]^T$ and $X=[\vec{x}_1,\vec{x}_2,...,\vec{x}_N]^T$.

Now, we are interested in estimating $\vec{\beta}$ from the observed data.
\begin{itemize}
	\item Derive the likelihood function $\mathcal{L}(\vec{\beta})$
	\item Show that the MLE estimator $\hat{\vec{\beta}}_{MLE}$ of $\vec{\beta}$ is equivalent to the solution of the following linear regression problem:
	\begin{equation}
	\label{eq:MLE}
	\min_{\vec{\beta}}\frac{1}{2}||\vec{Y}-X\vec{\beta}||_2^2
	\end{equation}
	\item Now we suppose 
\end{itemize}


\newpage

\section*{Part B, Problem 3: MLE, MAP and Logistic Regression}
%% Your answers here
\newpage

\section*{Part C: Programming Exercise}
%% Your answers here
\newpage


\end{document}
